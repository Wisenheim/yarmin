\chapter{ Desing Choice }


As a starting point, I used the denotation interpreter.
I chose to develop a denotational interpreter since adding
new constructs are simpler than the operative interpreter.\newline

The interpreter is characterized by a dynamic local environment and a static non-local environment.\newline

To simplify writing code from terminal to ocaml, I used rlwrap.\newline
The interpreter therefore consists of a series of functions divided into several files.

To start the program:
\newline
\textbf{1.} Get in the Path of the project directory;\newline
\textbf{2.} Run the script \textbf{compileAll.sh} to compile.\newline
\textbf{3.} Run the command  \textbf{rlwrap ocaml}\newline
\textbf{4.} Insert  \textbf{\#use "loadAll.ml";;}\newline



In the development of the reflection operation, I decided to write a
parser via \textbf{Menhir}, which is the evolution of the previous Ocamlyacc parser for OCaml.
Using it together with the use of ocamllex, or lexer, I have written a grammar
for the language so that I can properly interpret correctly any string passed as an argument of the reflect function and so on assigning the correct semantic evaluation.

All various test, can be executed by the command \textbf{\#use "YarminTest.ml";;}

\newpage
\subsection{ Files }
\begin{itemize}
\item \textbf{loadAll.ml}: Used for run the program.\newline
\item \textbf{Funstore.mli}: Store Interface .\newline
\item \textbf{Funstore.ml}: Store function implementation.\newline
\item \textbf{Funenv.mli}: Environment Interface.\newline
\item \textbf{Funenv.ml}: Environment function implementation.\newline
\item \textbf{YarminParser.mly}: Parser definition for Menhir.\newline
\item \textbf{YarminLexer.mll}: Lexer definition for OCamllex.\newline
\item \textbf{Yarmin.mli}: General Interface with types.\newline
\item \textbf{Yarmin.ml}: General Implementation of the language.\newline

\end{itemize}


\subsection{ Functions }
	\textbf{Semantic Valutation}\newline
\begin{itemize}

	\item SEM \newline
	\item SEMDEN \newline
	\item SEMC \newline
	\item SEMDV \newline
	\item SEMDR \newline
	\item SEMB \newline
\end{itemize}
	\newpage
	\textbf{Implementations}\newline
\begin{itemize}
	\item LEN \newline
	\item CAT \newline
	\item SUBSTR \newline
	\item EREFLECT \newline
\end{itemize}	

